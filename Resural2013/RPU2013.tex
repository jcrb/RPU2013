\documentclass[12pt,english,french,twoside]{report}
\usepackage[francais]{babel}
\usepackage[T1]{fontenc}
\usepackage{lmodern}
\usepackage[utf8]{inputenc}
% inclu un fichier de bibliothèques
\include{rpu-2012.sty}

% la directive suivante permet d'utiliser le séparateur de milliers. 
%Deux formulation sont utilisables mais exclusives l'une de l'autre. L'instruction "autolanguage" permet d'utiliser \nombre{} ou \np{}
% exemple: L'ensemble des SU ont déclaré \np{length(e)} passages en 2013
%\usepackage{numprint}
\usepackage[autolanguage,np]{numprint}

% la ligne qui suit donne un titre interne au document pdf et crée des liens cliquables en bleu sur les tables et figures
\usepackage[pdftitle={RPU2012-Resural}, colorlinks=true, linkcolor=blue,citecolor=blue, urlcolor=blue, linktocpage=true, breaklinks=true]{hyperref}

\usepackage[left=4cm,right=3cm,top=2cm,bottom=2cm]{geometry}
\usepackage{graphicx}

\usepackage{amsmath}
\usepackage{amsfonts}
\usepackage{amssymb}
\usepackage{xcolor} 
\usepackage{hyperref} 
\usepackage{tikz} 
% Mise en page
% LE zone gauche page paire
% CE zone médiane page paire
% RE zone droite page paire
% LO zone gauche page impaire
% CO zone médiane page impaire
% RO zone droite page impaire
% fancyhead gère le haut de page
% fancyfoot gère le bas de page
\usepackage{fancyhdr}
\pagestyle{fancy}
\fancyhead[LE,CE,RE,LO,CO,RO]{}
\fancyhead[LE,RO]{\scshape\thepage}
\fancyhead[CE]{\scshape\leftmark}
\fancyhead[CO]{\scshape\rightmark}

\fancyfoot[CE]{Document de travail - non validé}
\fancyfoot[CO]{Document de travail - non validé}

\usepackage{fancyvrb}
\usepackage{longtable}
\usepackage{lscape}
\usepackage{multirow}
\usepackage{array}
\usepackage{tabularx}

\usepackage{makeidx}
\makeindex

\bibliographystyle{plain}

\makeglossary
%\makeindex -s glossaire.ist RPU2013.glo -o RPU2013.glx

% \newenvironment{leglossaire}{\begin{list}{}{%
%   \setlength{\labelwidth}{.5\textwidth}%
%   \setlength{\labelsep}{-.8\labelwidth}%
%   \setlength{\itemindent}{\parindent}%
%   \setlength{\leftmargin}{25pt}%
%   \setlength{\rightmargin}{0pt}%
%   \setlength{\itemsep}{.8\baselineskip}%
%   \renewcommand{\makelabel}[1]{\boiteentreeglossaire{##1}}}}
% {\end{list}}
%       
% \newcommand{boiteentreeglossaire}[1]{%
%   \parbox[b]{\labelwidth}{%
%   \setlength{\fboxsep}{3pt}%
%   \setlength{\fboxrule}{.4pt}%
%   \shadow{\sffamily#1}\\\hfill\mbox{}}}

% Définitions de constantes



% Commandes
\providecommand{\tabularnewline}{\\} % compatibilité avec Lyx

\begin{document}

\title{Analyse des données RPU 2013 de la région Alsace}
\author{RESURAL\thanks{Réseau des urgences en Alsace - Equipe de coordination Dr J.C.
Bartier \& Madame Christine Hecker}}
\date{\today}
\maketitle

%\SweaveOpts{concordance=TRUE}
%\SweaveOpts{fig.path='./figure/rpu2013-', comment=NA, prompt=FALSE}

% page de garde n°2
%------------------
\newpage
\chapter*{}
\vfill
\begin{itemize}\raggedright
  \item R version 3.0.2 (2013-09-25), \verb|x86_64-redhat-linux-gnu|
  \item Locale: \verb|LC_CTYPE=fr_FR.UTF-8|, \verb|LC_NUMERIC=C|, \verb|LC_TIME=fr_FR.UTF-8|, \verb|LC_COLLATE=fr_FR.UTF-8|, \verb|LC_MONETARY=fr_FR.UTF-8|, \verb|LC_MESSAGES=fr_FR.UTF-8|, \verb|LC_PAPER=fr_FR.UTF-8|, \verb|LC_NAME=C|, \verb|LC_ADDRESS=C|, \verb|LC_TELEPHONE=C|, \verb|LC_MEASUREMENT=fr_FR.UTF-8|, \verb|LC_IDENTIFICATION=C|
  \item Base packages: base, datasets, graphics, grDevices,
    methods, stats, utils
  \item Other packages: knitr~1.5
  \item Loaded via a namespace (and not attached): evaluate~0.5.1,
    formatR~0.10, stringr~0.6.2, tools~3.0.2
\end{itemize}


% Making a derivative work?
% You are encouraged to leave this page entirely intact.
\noindent $\copyright$ RESURAL 2013. This content is available under a Creative Commons Attribution-ShareAlike 3.0 Unported United States license. License details are available at the Creative Commons website: \url{http://www.creativecommons.org} \\

\noindent For license and attribution guidance, see \url{http://www.openintro.org/perm/stat2nd_v2.txt}


\tableofcontents
\listoftables
\listoffigures

\begin{knitrout}
\definecolor{shadecolor}{rgb}{0.969, 0.969, 0.969}\color{fgcolor}\begin{kframe}


{\ttfamily\noindent\bfseries\color{errorcolor}{\#\# Error: there is no package called 'stargazer'}}

{\ttfamily\noindent\bfseries\color{errorcolor}{\#\# Error: impossible d'ouvrir la connexion}}

{\ttfamily\noindent\bfseries\color{errorcolor}{\#\# Error: there is no package called 'sp'}}\end{kframe}
\end{knitrout}

%======================================================== PARTIE 1
\part{Le Réseau des urgences en Alsace}
\newpage

\chapter{Historique}

% historique.Rnw

Le Réseau des Urgences en Alsace a été créé en août 2008 sous forme d'une association de droit local dans la foulée de la circulaire de 2007.

\index{RESURAL!historique}


\newpage
\chapter{Organisation géographique}






































